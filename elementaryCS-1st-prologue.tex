\section{ガイダンス}
%
%%% GOAL
%
\subsection{科目の概要}
\begin{frame}
\frametitle{科目の概要}
  \begin{itemize}
\item 学期: 水曜日 3-4 限
\item 場所: Zoom
\item 担当教員: 永藤 直行 (ナガトウ ナオユキ)
\item 連絡先: nagatou@presystems.xyz
\item 質問時間: 講義終了後かメイルで
\item CS4b クラスのサイト: OCW-i か \href{https://sites.google.com/a/presystems.xyz/sample/home/elementary-computer-science}{\beamerbutton{https://sites.google.com/a/presystems.xyz/sample/home/elementary-computer-science}} 
\item 共通サイト: \href{http://www.edu.gsic.titech.ac.jp/}{\beamerbutton{教育用電子計算機システム}}のサイトから辿ってください
\item その他ツールは必要に応じて URL を示します
  \end{itemize}
\end{frame}
\begin{frame}
\frametitle{TA の方々}
  \begin{itemize}
\item TA の方の自己紹介
  \end{itemize}
\end{frame}
%
%%% REFERENCE
%
%\subsection{参考図書}
%\begin{frame}
%\frametitle{参考図書}
%  \begin{itemize}
%\item 参考図書: 渡辺治 著,コンピュータサイエンス\textendash 計算を通して世界を観る, 丸善出版 (2015)
%\item 参考図書のサイト: \href{http://www.is.titech.ac.jp/~watanabe/csbook/}{\beamerbutton{http://www.is.titech.ac.jp/$\sim$watanabe/csbook/}} 
%  \end{itemize}
%\centering
%\includegraphics[scale=.25]{./Figure/TextBook.jpg}
%\end{frame}
%
%%% EVALUATION
%
\subsection{評価基準}
\begin{frame}
\frametitle{評価基準}
  \begin{itemize}
\item 講義は全 7 回,期末試験は行いません
\item 宿題: 3 回 \(3\times 5=15\) 点
%\item 課題: 4 回 \(15+15+20=50\) 点
\item 課題: 4 回 \(20+20+20+25=85\) 点
\item 宿題・課題提出: 
    \begin{itemize}
\item 講義時間中に課題を出します
\item 提出方法はその都度指定します
    \end{itemize}
\item 宿題と課題提出で出欠確認に変えます
  \end{itemize}
\end{frame}
%
%%% 3rd Quarter Schedule
%
\subsection{講義スケジュール}
\begin{frame}
\frametitle{講義日程}
%  \begin{itemize}
%\item 教室を間違えないように
%\item 進捗によってはまとめと試験は同一日に実施するかも
%  \end{itemize}
  \begin{center}
    \begin{tabular}{rll}
回&題目&場所\\
\hline
1&ガイダンス,環境構築,計算の基礎& Zoom\\
2&プログラミング演習&  Zoom\\
3&配列,文字列& Zoom\\
4&プログラミング演習& Zoom\\
5&暗号入門& Zoom\\
6&プログラミング演習& Zoom\\
7&まとめ& Zoom
%7&試験& 西 2 号館 W631
    \end{tabular}
  \end{center}
\end{frame}
