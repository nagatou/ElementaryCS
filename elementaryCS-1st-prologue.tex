\section{ガイダンス}
%
%%% GOAL
%
\subsection{科目の概要}
\begin{frame}
\frametitle{科目の概要}
  \begin{itemize}
\item 学期: 水曜日 3-4 限
\item 場所: Zoom
\item 担当教員: 永藤 直行 (ナガトウ ナオユキ)
\item 連絡先: nagatou@presystems.xyz
\item 質問時間: メイルか T2Schola の質問用チャットで
\item CS4b クラスのサイト: \href{https://sites.google.com/presystems.xyz/elementarycs/top}{\beamerbutton{https://sites.google.com/presystems.xyz/elementarycs/top}} 
\item 共通サイト: \href{https://wakita.github.io/classes/y21/cs1/course.html}{\beamerbutton{https://wakita.github.io/classes/y21/cs1/course.html}}
\item その他ツールは必要に応じて URL を示します
  \end{itemize}
\end{frame}
%\begin{frame}
%\frametitle{TA の方々}
%  \begin{itemize}
%\item TA の方の自己紹介
%  \end{itemize}
%\end{frame}
%
%%% REFERENCE
%
\subsection{参考図書}
\begin{frame}
\frametitle{参考図書}
  \begin{itemize}
\item 渡辺治 著,コンピュータサイエンス\textendash 計算を通して世界を観る, 丸善出版 (2015)
\item Kenneth H. Rosen 著, Discrete Mathematics and Its Applications 8th ed.(2018)
  \end{itemize}
  \begin{columns}[t]
    \begin{column}{0.45\textwidth}
\centering
\includegraphics[scale=.25]{./Figure/TextBook.jpg}
    \end{column}
    \begin{column}{0.45\textwidth}
\centering
\includegraphics[scale=.07]{./Figure/DMA.jpg}
    \end{column}
  \end{columns}
\end{frame}
%
%%% EVALUATION
%
\subsection{評価基準}
\begin{frame}
\frametitle{評価基準}
  \begin{itemize}
\item 講義は全 7 回,期末試験は行いません
\item 宿題: 3 回 くらい(提出不要)
\item 課題: 4 回 \(25+30+30=85\) 点
\item 特別課題: 1 回 \(15\) 点(提出任意)
\item 宿題・課題提出: 
    \begin{itemize}
\item 講義時間中に課題を出します
\item 提出方法はその都度指定します
    \end{itemize}
\item 宿題と課題提出で出欠確認に変えます
  \end{itemize}
\end{frame}
%
%%% 3rd Quarter Schedule
%
%\subsection{講義スケジュール}
%\begin{frame}
%\frametitle{講義日程}
%  \begin{itemize}
%\item 教室を間違えないように
%\item 進捗によってはまとめと試験は同一日に実施するかも
%  \end{itemize}
%  \begin{center}
%    \begin{tabular}{rll}
%回&題目&場所\\
%\hline
%1&ガイダンス,環境構築,計算の基礎& Zoom\\
%2&プログラミング演習&  Zoom\\
%3&配列,文字列& Zoom\\
%4&プログラミング演習& Zoom\\
%5&暗号入門& Zoom\\
%6&プログラミング演習& Zoom\\
%7&まとめ& Zoom
%%7&試験& 西 2 号館 W631
%    \end{tabular}
%  \end{center}
%\end{frame}
