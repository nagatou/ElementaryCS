\section{ビットとバイト}
\subsection{Bit とは}
\begin{frame}
\frametitle{情報とは}
  \begin{itemize}
\item ここで情報とは何か考えてみます
\item 情報とは"ある物事,事情についてのお知らせ"
\item 情報の価値はどう決まるか?
    \begin{itemize}
\item 驚きをもって受け止められる情報は価値が高い?
\item 日常的な情報は価値が低い?
    \end{itemize}
  \end{itemize}
\end{frame}
\begin{frame}
\frametitle{まずは情報量というもの}
  \begin{enumerate}
\item ある結果や情報を得る場合を考える
\item 結果や情報を生じる事象が確率現象であると見なす
\item 確率 $p$ の事象の情報量を \(I(p)\) であらわす
\item \(I(p)\) は単調減少関数
    \begin{itemize}
\item 頻繁に起こっていること ((\(p\) が大きい) は情報量が少ない)
\item 頻繁に起こらないこと ((\(p\) が小さい) は情報量が多い)
    \end{itemize}
\item 2 つの事象は独立
    \begin{itemize}
\item \(I(p_1~p_2)=I(p_1)+I(p_2)\)
    \end{itemize}
\item 連続関数である
    \begin{itemize}
\item 確率のわずかな変化で情報量が大きく変化するのは不自然
    \end{itemize}
  \end{enumerate}
  \begin{block}{情報量の定義}
ある事象 $a$ の生起確率を \(p_a\) とすると
その情報量 \(I(p_a)\) は \(\log_{2}\frac{1}{p_a}=-\log_{2}p_a\) であらわすことにする
  \end{block}
\end{frame}
\begin{frame}
\frametitle{ビットとは}
  \begin{block}{1 bit とは}
    \begin{enumerate}
\item 今, 2 つの事象を考える
\item 同じ確率 \(P=\frac{1}{2}\) で生起するとする
\item このとき,ひとつの事象 $a$ の情報量 \(I(a)=\log_2\frac{1}{\frac{1}{2}}=1\)
\item これが 1 bit
\item 確率 \(\frac{1}{2}\) で起こる事象を知った時の情報量が 1 bit(ビット)
    \end{enumerate}
  \end{block}
  \begin{itemize}
\item それでは確率 \(\frac{1}{10}\) で起こる事象を知った時は 1 hartley(ハートレー)
\item 確率 \(\frac{1}{e}\) では 1 nat(ナット)
  \end{itemize}
\end{frame}
\begin{frame}
\frametitle{情報の記録}
  \begin{itemize}
\item 情報はビットの列として記録
\item 明確に区別された 2 つの状態で記録しています
    \begin{itemize}
\item 磁性体の向き,電圧の高低,スイッチの開閉
    \end{itemize}
\item 計算機科学では 2 つの状態を便宜的に $0$ と $1$ として議論しています
  \end{itemize}
\end{frame}
\begin{frame}
\frametitle{ビットによる情報の表現}
  \begin{itemize}
\item 2 つの状態を取り得るデバイスを $N$ 個並べてそれぞれ独立としたらどれだけの情報があらわせるか
\item 答えは \(\log_2 2^N=N\) となり $N$ ビットの情報量となります
\item $N$ ビットでどれだけの事象を区別できるでしょうか
\item 答えは 2 個の要素から $N$ 個の重複順列 \({ }_2\Pi_N=2^N\) です
  \end{itemize}
\end{frame}
\subsection{Byte とは}
\begin{frame}
\frametitle{バイトとは}
  \begin{itemize}
\item 人間にとって意味をなす長さ $N$ の小ブロック
\item 現在のコンピュータでは 8 bit としています
  \end{itemize}
\end{frame}
\begin{frame}
\frametitle{8 ビットの由来}
  \begin{itemize}
\item ASCII コードを策定する時まで遡ります
\item 当時,電信で利用しているコードがありました
    \begin{itemize}
\item 電信では 5 ビットや 6 ビット
    \end{itemize}
\item 一方でコンピュータによるデータ処理が立ち上がっていました
    \begin{itemize}
\item EDSAC は 5 ビット
\item IBM 7030 が 8 ビット
\item パンチカードは 12 ビット: 12 rows$\times$80(-90) columns (パンチカードの写真参照)
    \end{itemize}
\item 電信,データ処理においてどれだけの文字を必要とするか
  \end{itemize}
\end{frame}
\begin{frame}
\frametitle{8 ビットの由来 -つづき}
  \begin{itemize}
\footnotesize
\item 電信で用いていた文字
    \begin{itemize}
\footnotesize
\item 1 列に 5 個まで穴を開けられる紙テープを使っていました(インターネット上の画像を参照)
\item アルファベット大文字と数字と特殊記号
    \end{itemize}
\item コンピュータが用いていた文字
    \begin{itemize}
\footnotesize
\item コンピュータの入出力に 12 ビットのパンチカードを使っていました
\item アルファベットの大文字 26 文字と数字(10個)といくつかの記号(11個)の 47 個あるいは 57 個 (UNIVAC の場合)
    \end{itemize}
\item データ処理では小文字も必要ということになって来ました.
\item 文字を処理するためにはさらに大文字,小文字を区別して 73(=47+26) 個; 7 ビット必要
    \begin{itemize}
\footnotesize
\item いくつの事象を区別したいかに依存します
\item アルファベットといくつかの記号だけならば \(2^7=128\) で足ります
\item でも,フランス語,ドイツ語などの文字と数学記号などを含めたら足りません
\item 漢字までは考えが及んでいたかどうかは分かりません
    \end{itemize}
  \end{itemize}
\end{frame}
\begin{frame}
\frametitle{8 ビットの由来 -つづき}
  \begin{itemize}
\footnotesize
\item 7 ビットで足りるが,拡張を考えて 8 ビットにすることを規格策定時に主張
    \begin{itemize}
\item 8 自身が \(2^3\) と二進数できりがいいこと
\item 4 ビットごとに分けられること
\item 1 ビット余るが,結果的には英語圏以外で空き領域を使っています
\item 漢字まで含めたら全く足りませんでした
    \end{itemize}
\item 当時 8 bit を 1 つのかたまりとして,これを バイト(byte) といっていました
\item それがそのまま定着して 1 byte = 8 bit と現在に至っています
\item 8 bit であることを強調してオクテット(octet) という人もいます
  \end{itemize}
\end{frame}
