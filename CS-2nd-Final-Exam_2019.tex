% subject: コンピュータサイエンス第一期末試験(雛形)
% date:    16/11/2
% LaTeX2e: Japanese

\documentclass[12pt]{article}
\pagestyle{empty}
\usepackage{ascmac}
\usepackage[dvips]{epsfig}
\usepackage{listings}

% local.mac
\lstset{language=,%
   basicstyle={\ttfamily\scriptsize},%
   lineskip=-0.5zw,%
   commentstyle=\textit,%
   classoffset=1,%
   keywordstyle=\bfseries,%
   breaklines=true,
   frame=lines,%
   framesep=5pt,%
   showstringspaces=false,%
   numbers=left,%
   numbersep=3pt,%
   stepnumber=1,%
   numberstyle={\ttfamily\scriptsize},%
%   numberstyle={\tiny},%
   numberblanklines=true,
   morekeywords={def, end, while, if, for, in, else, then, return},%
   escapechar=\@,
}%

%%% STYLE PARAM.s (for A4) %%%
\textwidth=16cm
\textheight=240mm

\topmargin=0mm
\headsep=0cm
\headheight=0cm
\oddsidemargin=0cm
\evensidemargin=0cm
\marginparwidth=0cm

\footnotesep=15pt
%\footheight=1.5cm
%\footskip=1.5cm

\itemsep=0.1pt
\parindent=11pt

\def\baselinestretch{1.15}

%%% LOCAL MACRO DEF.s %%%


% Print control (skips)
%\newcommand{\bigskip}{\vskip12pt}
%\newcommand{\medskip}{\vskip6pt}
%\newcommand{\smallskip}{\vskip3pt}
\newcommand{\paragraphskip}{\vskip\topsep}

% Itemization, etc
\newcommand{\nitem}[1]{%
\par\noindent\hangindent20pt%
\hskip20pt\llap{#1~}}
\newcommand{\nnitem}[1]{%
\par\noindent\hangindent30pt%
\hskip30pt\llap{#1~}}
\renewcommand{\labelenumi}{(\arabic{enumi})}

\begin{document}
\noindent
\hfill{\small 29.Jan.2019}

\noindent
\hfil
{\large\bf
コンピュータサイエンス第 2\textemdash 期末試験 4b(CS)\textemdash}
\hfil

\paragraphskip\noindent
※答案用紙は各問ごとに 1 枚使用して書くこと.\\
※答案用紙には各枚ごとに学籍番号と氏名を書くこと.\\
※計算の過程も解答用紙に残すこと.\\
※問題は裏面にもあります.


\paragraphskip
\nitem{\bf 問 1.}(配点 10 点)\\
プログラミング言語 Ruby で書かれたつぎの 2 つの関数 (a), (b) はどちらも引数を 1 増やす関数 succ と 1 減らす関数 pred を使って,
 2 つの自然数を足す方法を定義している.
以下の各問に答えよ.
  \begin{lstlisting}[name=add]
def succ (x)
  return x+1
end
def pred (x)
  return x-1
end
######
# (a) 
######
def add (a,b)
  if (a==0) then
    return b
  else
    return succ(add(pred(a),b))
  end
end
######
# (b)
######
def add_iter (a,b)
  if (a==0) then
    return b
  else
    return add_iter(pred(a),succ(b))
  end
end
  \end{lstlisting}
  \begin{enumerate}
\item add(4,5) あるいは add\_iter(4,5) としたときの (a),(b) それぞれの生成プロセスを講義スライド 36 ページのようにして示せ.
\item 2 つの方法の違いについて``計算状態''の用語を用いて説明せよ.
  \end{enumerate}
\paragraphskip
\nitem{\bf 問 2.}(配点 10 点)\\
つぎの各問に答えよ.計算の過程も解答用紙に残すこと.
\((n)_{m}\) は $n$ が $m$ 進表記であることを表すものとする.
  \begin{enumerate}
\item \((0.1)_{10}\) を 2 進表記に変換せよ.
\label{q:first}
\item (\ref{q:first}) で求めた数を 32 bits の浮動小数点数で表わせ.
ただし,符号に 1 bit,指数に 8 bits,仮数に 23 bits とする.
IEEE 754 に従って指数部は下駄をはかせること.
\label{q:secound}
\item (\ref{q:secound}) で求めた浮動小数点数は誤差を含む.この誤差を何というか.
  \end{enumerate}
\newpage
\paragraphskip
\nitem{\bf 問 3.}(配点 10 点)\\
以下はプログラミング言語 Ruby で書かれた 2 つ関数 (a), (b) について以下の問いに答えよ.
  \begin{lstlisting}[name=gcd]
#####
# (a)
#####
def min(x,y)
  if (x>y)
    return y
  else
    return x
  end
end
def gcd(x,y)
  ans=1
  n=min(x,y)
  for i in 1..n
    if (x%i==0)&&(y%i==0)
      ans=i
    end
  end
  return (ans)
end
#####
# (b)
#####
def fast_gcd(x1,x2)
  if (x2==0)
    return(x1)
  else
    return(fast_gcd(x2,x1%x2))
  end
end

  \end{lstlisting}
  \begin{enumerate}
\item (a), (b) は何をする関数か答えよ.
\item (a), (b) について,計算量を Big-O 記法で示せ.
  \end{enumerate}



\end{document}
