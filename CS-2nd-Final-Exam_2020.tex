% subject: コンピュータサイエンス第一期末試験(雛形)
% date:    16/11/2
% LaTeX2e: Japanese

\documentclass[12pt]{article}
\pagestyle{empty}
\usepackage{ascmac}
\usepackage[dvips]{epsfig}
\usepackage{listings}

% local.mac
\lstset{language=python,%
   basicstyle={\ttfamily\scriptsize},%
   lineskip=-0.5zw,%
   commentstyle={\ttfamily\scriptsize},%
   classoffset=1,%
   keywordstyle=\bfseries,%
   breaklines=true,
   frame=lines,%
   framesep=5pt,%
   showstringspaces=false,%
   numbers=left,%
   numbersep=3pt,%
   stepnumber=1,%
   numberstyle={\ttfamily\scriptsize},%
   numberblanklines=true,
   escapechar=\@,
}%

%%% STYLE PARAM.s (for A4) %%%
\textwidth=16cm
\textheight=240mm

\topmargin=0mm
\headsep=0cm
\headheight=0cm
\oddsidemargin=0cm
\evensidemargin=0cm
\marginparwidth=0cm

\footnotesep=15pt
%\footheight=1.5cm
%\footskip=1.5cm

\itemsep=0.1pt
\parindent=11pt

\def\baselinestretch{1.15}

%%% LOCAL MACRO DEF.s %%%


% Print control (skips)
%\newcommand{\bigskip}{\vskip12pt}
%\newcommand{\medskip}{\vskip6pt}
%\newcommand{\smallskip}{\vskip3pt}
\newcommand{\paragraphskip}{\vskip\topsep}

% Itemization, etc
\newcommand{\nitem}[1]{%
\par\noindent\hangindent20pt%
\hskip20pt\llap{#1~}}
\newcommand{\nnitem}[1]{%
\par\noindent\hangindent30pt%
\hskip30pt\llap{#1~}}
\renewcommand{\labelenumi}{(\arabic{enumi})}

\begin{document}
\noindent
\hfill{\small 29.Jan.2020}

\noindent
\hfil
{\large\bf
コンピュータサイエンス第 2\textemdash 期末試験 4b(CS)\textemdash}
\hfil

\paragraphskip\noindent
※答案用紙は各問ごとに 1 枚使用して書くこと.\\
※答案用紙には各枚ごとに学籍番号と氏名を書くこと.\\
※計算の過程も解答用紙に残すこと.\\
※問題は裏面にもあります.\\
※講義スライドは最新のものを利用してください.


\paragraphskip
\nitem{\bf 問 1.}(配点 15 点)\\
プログラミング言語 Python で書かれたつぎの関数 (a), (b) について以下の各問に答えよ.
  \begin{lstlisting}[name=power]
def square(x):
  return(x*x)
def is_even(n):
  if (n%2==0):
    return(True)
  else:
    return(False)
#####
# (a)
#####
def fast_power(b,n):
  if (n==1):
    return(b) 
  else:
    if (is_even(n)):
      return(square(fast_power(b,n/2)))
    else:
      return((b*fast_power(b,n-1)))
#####
# (b)
#####
def fast_power_cps(b,n):
  def fast_power_cps1(b,n,product):
    if (n==0):
      return(product) 
    else:
      if (is_even(n)):
        return(fast_power_cps1(square(b),n/2,product))
      else:
        return((fast_power_cps1(b,n-1,product*b)))
  return(fast_power_cps1(b,n,1))
  \end{lstlisting}
  \begin{enumerate}
\item プログラム (a), (b) はいずれも同じアルゴリズムを用いているが計算の様子の違う.それぞれの様子を \(2^{5}\) を求めるものとして講義スライド 33 ページの図のように示せ.
\item プログラム (a), (b) について 2 つの違いを``活性レコード''と``計算状態''の用語を用いて説明せよ.
\item プログラム (b) のような再帰を何と呼ぶか答えよ.
\item プログラム (a) について,\(b^{21}\)としたときの掛け算の計算回数を示せ.
\item 計算回数を一般化した式を示せ.
\item さらに,一般化した式を用いて時間計算量を Big-O 記法で示せ.
  \end{enumerate}
\newpage
\paragraphskip
\nitem{\bf 問 2.}(配点 15 点)\\
つぎの各問に答えよ.計算の過程も解答用紙に残すこと.
\((n)_{m}\) は $n$ が $m$ 進表記であることを表すものとする.
  \begin{enumerate}
\item \((0.1)_{10}\) を 2 進表記に変換せよ.
\label{q:first}
\item (\ref{q:first}) で求めた数を 32 bits の浮動小数点数で表わせ.
ただし,符号に 1 bit,指数に 8 bits,仮数に 23 bits とする.
IEEE 754 に従って指数部は下駄をはかせること.
\label{q:secound}
\item (\ref{q:secound}) で求めた浮動小数点数は誤差を含む.この誤差を何というか.
  \end{enumerate}


\end{document}
