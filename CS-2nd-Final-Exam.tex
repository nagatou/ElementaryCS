% subject: コンピュータサイエンス第一期末試験(雛形)
% date:    16/11/2
% LaTeX2e: Japanese

\documentclass[12pt]{article}
\pagestyle{empty}
\usepackage{ascmac}
\usepackage[dvips]{epsfig}
\usepackage{listings}

% local.mac
\lstset{language=,%
   basicstyle={\ttfamily\scriptsize},%
   lineskip=-0.5zw,%
   commentstyle=\textit,%
   classoffset=1,%
   keywordstyle=\bfseries,%
   breaklines=true,
   frame=lines,%
   framesep=5pt,%
   showstringspaces=false,%
   numbers=left,%
   numbersep=3pt,%
   stepnumber=1,%
   numberstyle={\ttfamily\scriptsize},%
%   numberstyle={\tiny},%
   numberblanklines=true,
   morekeywords={def, end, while, if, for, in, else, then, return},%
   escapechar=\@,
}%

%%% STYLE PARAM.s (for A4) %%%
\textwidth=16cm
\textheight=240mm

\topmargin=0mm
\headsep=0cm
\headheight=0cm
\oddsidemargin=0cm
\evensidemargin=0cm
\marginparwidth=0cm

\footnotesep=15pt
%\footheight=1.5cm
%\footskip=1.5cm

\itemsep=0.1pt
\parindent=11pt

\def\baselinestretch{1.15}

%%% LOCAL MACRO DEF.s %%%


% Print control (skips)
%\newcommand{\bigskip}{\vskip12pt}
%\newcommand{\medskip}{\vskip6pt}
%\newcommand{\smallskip}{\vskip3pt}
\newcommand{\paragraphskip}{\vskip\topsep}

% Itemization, etc
\newcommand{\nitem}[1]{%
\par\noindent\hangindent20pt%
\hskip20pt\llap{#1~}}
\newcommand{\nnitem}[1]{%
\par\noindent\hangindent30pt%
\hskip30pt\llap{#1~}}
\renewcommand{\labelenumi}{(\arabic{enumi})}

\begin{document}
\noindent
\hfill{\small 8.Feb.2017}

\noindent
\hfil
{\large\bf
コンピュータサイエンス第 2\textemdash 期末試験 CS4b\textemdash}
\hfil

\paragraphskip\noindent
※答案用紙は各問ごとに 1 枚使用して書くこと.\\
※答案用紙には各枚ごとに学籍番号と氏名を書くこと.


\paragraphskip
\nitem{\bf 問1.}(配点 10 点)\\
つぎの問に答えよ.計算の過程も解答用紙に残すこと.
\((n)_{m}\) は $n$ が $m$ 進表記であることを表すものとする.
  \begin{enumerate}
\item \((0.1)_{10}\) を 32 bits の浮動小数点数に変換せよ.符号に 1 bit,指数に 8 bits,仮数に 23 bits とする.
ただし,規格に従って指数部は下駄をはかせること.
\item \((5.005)_{10}+(1.00003)_{10}\) を 10 進で 5 桁まで記憶できるコンピュータで計算したときの丸め誤差を求めよ
(2 進数に変換せず,10 進のままで考えてよい).
\item つぎの無限級数の和と第 n 項までの和の差の絶対値は第 n+1 項の絶対値より小さくなる. \(x=1.0\) として
第 5 項で打ち切ったときの打ち切り誤差の絶対値の上限を求めよ.
ここで \(n=0,1,\cdots\) とする.
 \[\sin(x)=x-\frac{x^3}{3!}+\frac{x^5}{5!}-\frac{x^7}{7!}\cdots+(-1)^{n}\frac{x^{2n+1}}{(2n+1)!}\cdots\]
  \end{enumerate}

\newpage
\paragraphskip
\nitem{\bf 問2.}(配点 10 点)\\
プログラミング言語 Ruby で書かれたプログラムの一部を以下に示す.
このプログラムについてつぎの問いに答えよ.
  \begin{lstlisting}[name=wholemodel]
def min_el(dis,rest)
  tmp=rest.first
  rest.each do |v|
    if(dis[tmp]>dis[v])
      tmp=v
    end
  end
  return(tmp)
end
def min(a,b)
  if(a<b)
    return(a)
  else
    return(b)
  end
end 
def dijkstra(start, destination, graph, p)
  d=Array.new(28)
  s=Array.new(28)
  rest=[0,1,2,3,4,5,6,7,8,9,10,11,12,13,14,15,16,17,18,19,20,21,22,23,24,25,26,27]
  for i in 0..27
    d[i]=graph[start][i]
  end
  s.push(start)
  rest.delete(start)
  n=rest.length
  i=0
  while i<n
    w=min_el(d,rest)
    s.push(w)
    rest.delete(w)
    rest.each do |v|
      if(d[w]+graph[w][v]<d[v])
        d[v]=d[w]+graph[w][v]
        @\fbox{ ~~~~~~~ (1) ~~~~~~}@
      end
    end
    i=i+1
  end
  return(p)
end
  \end{lstlisting}
  \begin{enumerate}
\item {\tt p} は最短経路を保持するための配列である.これをもちいて $\fbox{(1)}$ の中を完成させよ.
\item {\tt rest.delete(e)} は配列 {\tt rest} から要素 {\tt e} を削除した配列を, 
 {\tt rest.each do $|$v$|$} は配列 {\tt rest} の各要素 {\tt v} についての繰り返しを意味している.
このとき,グラフのノード数 $n$ として 17 行-41 行までの時間計算量を求めよ.
  \end{enumerate}

\newpage
\paragraphskip
\nitem{\bf 問3.}(配点 10 点)\\
プログラミング言語 Ruby で書かれたプログラムの一部を以下に示す.
このプログラムについてつぎの問いに答えよ.
  \begin{lstlisting}
def eps_m ()
  epsilon=1.0
  old=0.0
  prod=0.0
  cnt=0
  while (prod!=1.0)
    old = epsilon
    cnt=cnt+1
    epsilon=epsilon/2.0
    prod=epsilon+1.0
  end
  return(old)
end
#
def f_iter (guess,n,eps,previous)
  if is_enough(guess,eps,previous) then
     return(guess)
  else
     return(sqrt_iter(improve(guess,n),n,eps,guess))
  end
end
def improve (guess,n)
  return((guess+(n/guess))/2.0)
end
def is_enough (guess,eps,previous)
  return(abs(previous-guess)<(2.0*eps))
end
def abs (x)
  if x<0 then
     return(-x)
  else
     return(x)
  end
end
def f1 (n,eps)
  return(f_iter(1.0,n,(2*eps),0.0))
end
def f (n)
  print(f1(n,eps_m()),"\n")
end
  \end{lstlisting}
  \begin{enumerate}
\item 漸化式で表せるものは自然に再帰プログラムに表せる.{\tt f\_iter()} を漸化式の形で示せ.
\item {\tt is\_enough} で数列は収束したと判定している.このときの収束判定条件を示せ.
\item {\tt eps\_m()} は何を求めている関数か述べよ.
\item このプログラムを用いて {\tt f(2.0)} を計算したときの各繰り返しでの {\tt guess} の値を示せ.
ただし, 10 進表記で 9 桁まで記憶できると仮定する.
\item 自然数 $n$ に対して,{\tt f(}$n${\tt)} は,どのような関数か述べよ.
  \end{enumerate}

\end{document}
