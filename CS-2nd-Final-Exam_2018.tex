% subject: コンピュータサイエンス第一期末試験(雛形)
% date:    16/11/2
% LaTeX2e: Japanese

\documentclass[12pt]{article}
\pagestyle{empty}
\usepackage{ascmac}
\usepackage[dvips]{epsfig}
\usepackage{listings}

% local.mac
\lstset{language=,%
   basicstyle={\ttfamily\scriptsize},%
   lineskip=-0.5zw,%
   commentstyle=\textit,%
   classoffset=1,%
   keywordstyle=\bfseries,%
   breaklines=true,
   frame=lines,%
   framesep=5pt,%
   showstringspaces=false,%
   numbers=left,%
   numbersep=3pt,%
   stepnumber=1,%
   numberstyle={\ttfamily\scriptsize},%
%   numberstyle={\tiny},%
   numberblanklines=true,
   morekeywords={def, end, while, if, for, in, else, then, return},%
   escapechar=\@,
}%

%%% STYLE PARAM.s (for A4) %%%
\textwidth=16cm
\textheight=240mm

\topmargin=0mm
\headsep=0cm
\headheight=0cm
\oddsidemargin=0cm
\evensidemargin=0cm
\marginparwidth=0cm

\footnotesep=15pt
%\footheight=1.5cm
%\footskip=1.5cm

\itemsep=0.1pt
\parindent=11pt

\def\baselinestretch{1.15}

%%% LOCAL MACRO DEF.s %%%


% Print control (skips)
%\newcommand{\bigskip}{\vskip12pt}
%\newcommand{\medskip}{\vskip6pt}
%\newcommand{\smallskip}{\vskip3pt}
\newcommand{\paragraphskip}{\vskip\topsep}

% Itemization, etc
\newcommand{\nitem}[1]{%
\par\noindent\hangindent20pt%
\hskip20pt\llap{#1~}}
\newcommand{\nnitem}[1]{%
\par\noindent\hangindent30pt%
\hskip30pt\llap{#1~}}
\renewcommand{\labelenumi}{(\arabic{enumi})}

\begin{document}
\noindent
\hfill{\small 31.Jan.2018}

\noindent
\hfil
{\large\bf
コンピュータサイエンス第 2\textemdash 期末試験 CS4b\textemdash}
\hfil

\paragraphskip\noindent
※答案用紙は各問ごとに 1 枚使用して書くこと.\\
※答案用紙には各枚ごとに学籍番号と氏名を書くこと.


\paragraphskip
\nitem{\bf 問1.}(配点 15 点)\\
つぎの各問に答えよ.計算の過程も解答用紙に残すこと.
\((n)_{m}\) は $n$ が $m$ 進表記であることを表すものとする.
  \begin{enumerate}
\item \((0.1)_{10}\) を 2 進表記に変換せよ.
\label{q:first}
\item (\ref{q:first}) で求めた数を 32 bits の浮動小数点数で表わせ.
ただし,符号に 1 bit,指数に 8 bits,仮数に 23 bits とする.
IEEE 754 に従って指数部は下駄をはかせること.
\label{q:secound}
\item (\ref{q:secound}) で求めた浮動小数点数は誤差を含む.この誤差を何というか.
  \end{enumerate}

\newpage
\paragraphskip
\nitem{\bf 問2.}(配点 15 点)\\
以下はプログラミング言語 Ruby で書かれたプログラムである.
関数 A, B, C は同じ計算を行う.
このプログラムについてつぎの問いに答えよ.
  \begin{lstlisting}[name=wholemodel]
require 'benchmark'
######
# A
######
def A(b,n)
  if (n==0)
    return 1
  else
    return(b * A(b, n-1))
  end
end
######
# B
######
def B_1(b,n,product)
  if (n==0)
    return product
  else
    return(B_1(b,n-1,(b*product)))
  end
end
def B(b,n)
  return B_1(b,n,1)
end
######
# C
######
def even?(n)
  if (n%2==0)
    return true
  else
    return false
  end
end
def square(x)
  return (x*x)
end
def C(b,n)
  if (n==1) # スライドと少し違うので注意
    return b 
  else
    if (even?(n))
      return square(C(b,n/2))
    else
      return (b*C(b,n-1))
    end
  end
end

# Test Harness
x = gets().to_i
y = gets().to_i
Benchmark.bm(8) do |tmp|
  tmp.report(" "){
    puts(A(x,y))
  }
end
Benchmark.bm(8) do |tmp|
  tmp.report(" "){
    puts(B(x,y))
  }
end
Benchmark.bm(8) do |tmp|
  tmp.report(" "){
    puts(C(x,y))
  }
end
  \end{lstlisting}
  \begin{enumerate}
\item 何を計算するプログラムか答えよ.
\item B と C について \(\tt{x}=2, \tt{y}=998\) のときの掛け算の回数を求めよ.
\item \(\tt{y}=n\) として,A, B, C の各関数について,演算回数の増減を Big-O 記法で示せ.
\item A, B には再帰的な計算過程に違いがある.この違いについて説明せよ.
  \end{enumerate}


\end{document}
