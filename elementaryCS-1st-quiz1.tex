%
%%% QUIZ 1
%
\section{課題 1 演習ガイド}
\subsection{課題 1 の説明}
\begin{frame}
\frametitle{課題 1の説明}
\framesubtitle{計算でアニメーション}
  \begin{itemize}
\item 課題: 計算でアニメーションを作成してください
    \begin{itemize}
\item 動きがあること
\item 計算だけで動かすこと
    \end{itemize}
\item 提出物:
    \begin{itemize}
\item 作成したアニメーションプログラムのソースコード
\item 作成したアニメーションプログラムの計算の仕組みの説明はソースコードに埋め込む
\item Python 言語の説明は不要
    \end{itemize}
\item 提出は T2Schola から Python ソースコードを提出
  \end{itemize}
\end{frame}
%\begin{frame}[fragile]
%\frametitle{四則演算でアニメーションの例}
%  \begin{itemize}
%\item 10 個の自然数
%\item 先頭は 1 にしないとずれる
%  \end{itemize}
%\tiny
%  \begin{columns}
%    \begin{column}{0.55\textwidth}
%      \begin{lstlisting}[caption={smile.py},label=smile-data]
%# smile.py
%# 入力: 10個の自然数
%# 出力: スマイルマーク
%
%d1 =1000000000000000000000000000
%d2 =1000000000110000110000000000
%d3 =1000000000110000110000000000
%d4 =1000000000000000000000000000
%d5 =1000001100000000000011000000
%d6 =1000000110000000000110000000
%d7 =1000000011000000001100000000
%d8 =1000000000111111110000000000
%d9 =1000000000000000000000000000
%d10=1000000000000000000000000000
%      \end{lstlisting}
%    \end{column}
%    \begin{column}{0.4\textwidth}
%      \begin{lstlisting}[caption={smile.py},label=smile-body,firstnumber=last]
%t = 0
%while t < 15
%  puts(t)
%  puts(d1)
%  puts(d2)
%  puts(d3)
%  puts(d4)
%  puts(d5)
%  puts(d6)
%  puts(d7)
%  puts(d8)
%  puts(d9)
%  puts(d10)
%  puts()   # 空行を出力
%  sleep(1) # 1秒休む
%  t = t + 1
%end
%      \end{lstlisting}
%    \end{column}
%  \end{columns}
%\end{frame}
%\begin{frame}[fragile]
%\frametitle{スマイルを動かしてみる}
%\scriptsize
%  \begin{itemize}
%\item ここでいう動かすとはデータを計算で変化させていってそれを描画
%\item 10 で割っているので一番右のけたが落ちていきます
%  \end{itemize}
%  \begin{lstlisting}[caption={smile2.py},label=smile-destroy]
%# smile2.py
%# 入力: 10個の自然数
%# 出力: スマイルマークが,どうなるか?
%...
%while t < 29
%...
%  d1 = d1 / 10
%  d2 = d2 / 10
%  d3 = d3 / 10
%  d4 = d4 / 10
%  d5 = d5 / 10
%  d6 = d6 / 10
%  d7 = d7 / 10
%  d8 = d8 / 10
%  d9 = d9 / 10
%  d10 = d10 / 10
%
%  t = t + 1
%end
%  \end{lstlisting}
%\end{frame}
\begin{frame}[fragile,shrink]
\frametitle{ひつじさんを動かしてみる}
  \begin{itemize}
\scriptsize
\item 大きな数字を複数定義して 1 枚の紙に見立てる
\item 各桁をひとつの画素とみなす
\item 各桁は 0\textendash 9 でこの違いで絵にする
\item ここでは 14 個の自然数
\item 先頭は 1 にしないとずれる
\item 動かすときは桁をシフトさせて (ここでは 10 で割っている) 動かす
  \end{itemize}
  \begin{lstlisting}[caption={sheep.py (declaration)},label=sheep-data]
################
# pattern def. #
################
d0  = 1000000000000000000000000000000000000000000000000000000000
d1  = 1000000000000000000000000011111111000000000000000000000000
d2  = 1000000000000000000000001110000000110000000000000000000000
d3  = 1000000011111111111110011000011000111111000000000000000000
d4  = 1000001110000000000000001100111000110000110000000000000000
d5  = 1000001100000000000000000111110011000100011000000000000000
d6  = 1000001100000000000000000000000000000001111000000000000000
d7  = 1000001100000000000000000000000000000111000000000000000000
d8  = 1000001100000000000000000000000000011100000000000000000000
d9  = 1000001100000000000000000000000000011000000000000000000000
d10 = 1000001100000000000000000000000000111000000000000000000000
d11 = 1000000011000110001111111000110001110000000000000000000000
d12 = 1000000000111001110000000111001110000000000000000000000000
d13 = 1000000000000000000000000000000000000000000000000000000000
  \end{lstlisting}
\end{frame}
\begin{frame}[fragile,shrink]
\frametitle{ひつじさんを動かしてみる}
  \begin{itemize}
\scriptsize
\item 動かすときは桁をシフトさせて (ここでは 10 で割っている) 動かす
\item {\tt d0} を足しているのは絵がずれないようにするため
\item より高度な例 \href{https://www.a1k0n.net/2011/07/20/donut-math.html}{\beamerbutton{https://www.a1k0n.net/2011/07/20/donut-math.html}}
  \end{itemize}
  \begin{lstlisting}[caption={sheep.py (shift)},label=sheep-shift,firstnumber=last]
      # shift #
      d1  = (d1 -  d0) // 10 + d0
      d2  = (d2  - d0) // 10 + d0
      d3  = (d3  - d0) // 10 + d0
      d4  = (d4  - d0) // 10 + d0
      d5  = (d5  - d0) // 10 + d0
      d6  = (d6  - d0) // 10 + d0
      d7  = (d7  - d0) // 10 + d0
      d8  = (d8  - d0) // 10 + d0
      d9  = (d9  - d0) // 10 + d0
      d10 = (d10 - d0) // 10 + d0
      d11 = (d11 - d0) // 10 + d0
      d12 = (d12 - d0) // 10 + d0
      d13 = (d13 - d0) // 10 + d0
  \end{lstlisting}
\end{frame}
